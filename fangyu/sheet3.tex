
% Default to the notebook output style

    


% Inherit from the specified cell style.




    
\documentclass{article}

    
    
    \usepackage{graphicx} % Used to insert images
    \usepackage{adjustbox} % Used to constrain images to a maximum size 
    \usepackage{color} % Allow colors to be defined
    \usepackage{enumerate} % Needed for markdown enumerations to work
    \usepackage{geometry} % Used to adjust the document margins
    \usepackage{amsmath} % Equations
    \usepackage{amssymb} % Equations
    \usepackage{eurosym} % defines \euro
    \usepackage[mathletters]{ucs} % Extended unicode (utf-8) support
    \usepackage[utf8x]{inputenc} % Allow utf-8 characters in the tex document
    \usepackage{fancyvrb} % verbatim replacement that allows latex
    \usepackage{grffile} % extends the file name processing of package graphics 
                         % to support a larger range 
    % The hyperref package gives us a pdf with properly built
    % internal navigation ('pdf bookmarks' for the table of contents,
    % internal cross-reference links, web links for URLs, etc.)
    \usepackage{hyperref}
    \usepackage{longtable} % longtable support required by pandoc >1.10
    \usepackage{booktabs}  % table support for pandoc > 1.12.2
    \usepackage{ulem} % ulem is needed to support strikethroughs (\sout)
    

    
    
    \definecolor{orange}{cmyk}{0,0.4,0.8,0.2}
    \definecolor{darkorange}{rgb}{.71,0.21,0.01}
    \definecolor{darkgreen}{rgb}{.12,.54,.11}
    \definecolor{myteal}{rgb}{.26, .44, .56}
    \definecolor{gray}{gray}{0.45}
    \definecolor{lightgray}{gray}{.95}
    \definecolor{mediumgray}{gray}{.8}
    \definecolor{inputbackground}{rgb}{.95, .95, .85}
    \definecolor{outputbackground}{rgb}{.95, .95, .95}
    \definecolor{traceback}{rgb}{1, .95, .95}
    % ansi colors
    \definecolor{red}{rgb}{.6,0,0}
    \definecolor{green}{rgb}{0,.65,0}
    \definecolor{brown}{rgb}{0.6,0.6,0}
    \definecolor{blue}{rgb}{0,.145,.698}
    \definecolor{purple}{rgb}{.698,.145,.698}
    \definecolor{cyan}{rgb}{0,.698,.698}
    \definecolor{lightgray}{gray}{0.5}
    
    % bright ansi colors
    \definecolor{darkgray}{gray}{0.25}
    \definecolor{lightred}{rgb}{1.0,0.39,0.28}
    \definecolor{lightgreen}{rgb}{0.48,0.99,0.0}
    \definecolor{lightblue}{rgb}{0.53,0.81,0.92}
    \definecolor{lightpurple}{rgb}{0.87,0.63,0.87}
    \definecolor{lightcyan}{rgb}{0.5,1.0,0.83}
    
    % commands and environments needed by pandoc snippets
    % extracted from the output of `pandoc -s`
    \providecommand{\tightlist}{%
      \setlength{\itemsep}{0pt}\setlength{\parskip}{0pt}}
    \DefineVerbatimEnvironment{Highlighting}{Verbatim}{commandchars=\\\{\}}
    % Add ',fontsize=\small' for more characters per line
    \newenvironment{Shaded}{}{}
    \newcommand{\KeywordTok}[1]{\textcolor[rgb]{0.00,0.44,0.13}{\textbf{{#1}}}}
    \newcommand{\DataTypeTok}[1]{\textcolor[rgb]{0.56,0.13,0.00}{{#1}}}
    \newcommand{\DecValTok}[1]{\textcolor[rgb]{0.25,0.63,0.44}{{#1}}}
    \newcommand{\BaseNTok}[1]{\textcolor[rgb]{0.25,0.63,0.44}{{#1}}}
    \newcommand{\FloatTok}[1]{\textcolor[rgb]{0.25,0.63,0.44}{{#1}}}
    \newcommand{\CharTok}[1]{\textcolor[rgb]{0.25,0.44,0.63}{{#1}}}
    \newcommand{\StringTok}[1]{\textcolor[rgb]{0.25,0.44,0.63}{{#1}}}
    \newcommand{\CommentTok}[1]{\textcolor[rgb]{0.38,0.63,0.69}{\textit{{#1}}}}
    \newcommand{\OtherTok}[1]{\textcolor[rgb]{0.00,0.44,0.13}{{#1}}}
    \newcommand{\AlertTok}[1]{\textcolor[rgb]{1.00,0.00,0.00}{\textbf{{#1}}}}
    \newcommand{\FunctionTok}[1]{\textcolor[rgb]{0.02,0.16,0.49}{{#1}}}
    \newcommand{\RegionMarkerTok}[1]{{#1}}
    \newcommand{\ErrorTok}[1]{\textcolor[rgb]{1.00,0.00,0.00}{\textbf{{#1}}}}
    \newcommand{\NormalTok}[1]{{#1}}
    
    % Additional commands for more recent versions of Pandoc
    \newcommand{\ConstantTok}[1]{\textcolor[rgb]{0.53,0.00,0.00}{{#1}}}
    \newcommand{\SpecialCharTok}[1]{\textcolor[rgb]{0.25,0.44,0.63}{{#1}}}
    \newcommand{\VerbatimStringTok}[1]{\textcolor[rgb]{0.25,0.44,0.63}{{#1}}}
    \newcommand{\SpecialStringTok}[1]{\textcolor[rgb]{0.73,0.40,0.53}{{#1}}}
    \newcommand{\ImportTok}[1]{{#1}}
    \newcommand{\DocumentationTok}[1]{\textcolor[rgb]{0.73,0.13,0.13}{\textit{{#1}}}}
    \newcommand{\AnnotationTok}[1]{\textcolor[rgb]{0.38,0.63,0.69}{\textbf{\textit{{#1}}}}}
    \newcommand{\CommentVarTok}[1]{\textcolor[rgb]{0.38,0.63,0.69}{\textbf{\textit{{#1}}}}}
    \newcommand{\VariableTok}[1]{\textcolor[rgb]{0.10,0.09,0.49}{{#1}}}
    \newcommand{\ControlFlowTok}[1]{\textcolor[rgb]{0.00,0.44,0.13}{\textbf{{#1}}}}
    \newcommand{\OperatorTok}[1]{\textcolor[rgb]{0.40,0.40,0.40}{{#1}}}
    \newcommand{\BuiltInTok}[1]{{#1}}
    \newcommand{\ExtensionTok}[1]{{#1}}
    \newcommand{\PreprocessorTok}[1]{\textcolor[rgb]{0.74,0.48,0.00}{{#1}}}
    \newcommand{\AttributeTok}[1]{\textcolor[rgb]{0.49,0.56,0.16}{{#1}}}
    \newcommand{\InformationTok}[1]{\textcolor[rgb]{0.38,0.63,0.69}{\textbf{\textit{{#1}}}}}
    \newcommand{\WarningTok}[1]{\textcolor[rgb]{0.38,0.63,0.69}{\textbf{\textit{{#1}}}}}
    
    
    % Define a nice break command that doesn't care if a line doesn't already
    % exist.
    \def\br{\hspace*{\fill} \\* }
    % Math Jax compatability definitions
    \def\gt{>}
    \def\lt{<}
    % Document parameters
    \title{sheet3}
    
    
    

    % Pygments definitions
    
\makeatletter
\def\PY@reset{\let\PY@it=\relax \let\PY@bf=\relax%
    \let\PY@ul=\relax \let\PY@tc=\relax%
    \let\PY@bc=\relax \let\PY@ff=\relax}
\def\PY@tok#1{\csname PY@tok@#1\endcsname}
\def\PY@toks#1+{\ifx\relax#1\empty\else%
    \PY@tok{#1}\expandafter\PY@toks\fi}
\def\PY@do#1{\PY@bc{\PY@tc{\PY@ul{%
    \PY@it{\PY@bf{\PY@ff{#1}}}}}}}
\def\PY#1#2{\PY@reset\PY@toks#1+\relax+\PY@do{#2}}

\expandafter\def\csname PY@tok@gd\endcsname{\def\PY@tc##1{\textcolor[rgb]{0.63,0.00,0.00}{##1}}}
\expandafter\def\csname PY@tok@gu\endcsname{\let\PY@bf=\textbf\def\PY@tc##1{\textcolor[rgb]{0.50,0.00,0.50}{##1}}}
\expandafter\def\csname PY@tok@gt\endcsname{\def\PY@tc##1{\textcolor[rgb]{0.00,0.27,0.87}{##1}}}
\expandafter\def\csname PY@tok@gs\endcsname{\let\PY@bf=\textbf}
\expandafter\def\csname PY@tok@gr\endcsname{\def\PY@tc##1{\textcolor[rgb]{1.00,0.00,0.00}{##1}}}
\expandafter\def\csname PY@tok@cm\endcsname{\let\PY@it=\textit\def\PY@tc##1{\textcolor[rgb]{0.25,0.50,0.50}{##1}}}
\expandafter\def\csname PY@tok@vg\endcsname{\def\PY@tc##1{\textcolor[rgb]{0.10,0.09,0.49}{##1}}}
\expandafter\def\csname PY@tok@vi\endcsname{\def\PY@tc##1{\textcolor[rgb]{0.10,0.09,0.49}{##1}}}
\expandafter\def\csname PY@tok@mh\endcsname{\def\PY@tc##1{\textcolor[rgb]{0.40,0.40,0.40}{##1}}}
\expandafter\def\csname PY@tok@cs\endcsname{\let\PY@it=\textit\def\PY@tc##1{\textcolor[rgb]{0.25,0.50,0.50}{##1}}}
\expandafter\def\csname PY@tok@ge\endcsname{\let\PY@it=\textit}
\expandafter\def\csname PY@tok@vc\endcsname{\def\PY@tc##1{\textcolor[rgb]{0.10,0.09,0.49}{##1}}}
\expandafter\def\csname PY@tok@il\endcsname{\def\PY@tc##1{\textcolor[rgb]{0.40,0.40,0.40}{##1}}}
\expandafter\def\csname PY@tok@go\endcsname{\def\PY@tc##1{\textcolor[rgb]{0.53,0.53,0.53}{##1}}}
\expandafter\def\csname PY@tok@cp\endcsname{\def\PY@tc##1{\textcolor[rgb]{0.74,0.48,0.00}{##1}}}
\expandafter\def\csname PY@tok@gi\endcsname{\def\PY@tc##1{\textcolor[rgb]{0.00,0.63,0.00}{##1}}}
\expandafter\def\csname PY@tok@gh\endcsname{\let\PY@bf=\textbf\def\PY@tc##1{\textcolor[rgb]{0.00,0.00,0.50}{##1}}}
\expandafter\def\csname PY@tok@ni\endcsname{\let\PY@bf=\textbf\def\PY@tc##1{\textcolor[rgb]{0.60,0.60,0.60}{##1}}}
\expandafter\def\csname PY@tok@nl\endcsname{\def\PY@tc##1{\textcolor[rgb]{0.63,0.63,0.00}{##1}}}
\expandafter\def\csname PY@tok@nn\endcsname{\let\PY@bf=\textbf\def\PY@tc##1{\textcolor[rgb]{0.00,0.00,1.00}{##1}}}
\expandafter\def\csname PY@tok@no\endcsname{\def\PY@tc##1{\textcolor[rgb]{0.53,0.00,0.00}{##1}}}
\expandafter\def\csname PY@tok@na\endcsname{\def\PY@tc##1{\textcolor[rgb]{0.49,0.56,0.16}{##1}}}
\expandafter\def\csname PY@tok@nb\endcsname{\def\PY@tc##1{\textcolor[rgb]{0.00,0.50,0.00}{##1}}}
\expandafter\def\csname PY@tok@nc\endcsname{\let\PY@bf=\textbf\def\PY@tc##1{\textcolor[rgb]{0.00,0.00,1.00}{##1}}}
\expandafter\def\csname PY@tok@nd\endcsname{\def\PY@tc##1{\textcolor[rgb]{0.67,0.13,1.00}{##1}}}
\expandafter\def\csname PY@tok@ne\endcsname{\let\PY@bf=\textbf\def\PY@tc##1{\textcolor[rgb]{0.82,0.25,0.23}{##1}}}
\expandafter\def\csname PY@tok@nf\endcsname{\def\PY@tc##1{\textcolor[rgb]{0.00,0.00,1.00}{##1}}}
\expandafter\def\csname PY@tok@si\endcsname{\let\PY@bf=\textbf\def\PY@tc##1{\textcolor[rgb]{0.73,0.40,0.53}{##1}}}
\expandafter\def\csname PY@tok@s2\endcsname{\def\PY@tc##1{\textcolor[rgb]{0.73,0.13,0.13}{##1}}}
\expandafter\def\csname PY@tok@nt\endcsname{\let\PY@bf=\textbf\def\PY@tc##1{\textcolor[rgb]{0.00,0.50,0.00}{##1}}}
\expandafter\def\csname PY@tok@nv\endcsname{\def\PY@tc##1{\textcolor[rgb]{0.10,0.09,0.49}{##1}}}
\expandafter\def\csname PY@tok@s1\endcsname{\def\PY@tc##1{\textcolor[rgb]{0.73,0.13,0.13}{##1}}}
\expandafter\def\csname PY@tok@ch\endcsname{\let\PY@it=\textit\def\PY@tc##1{\textcolor[rgb]{0.25,0.50,0.50}{##1}}}
\expandafter\def\csname PY@tok@m\endcsname{\def\PY@tc##1{\textcolor[rgb]{0.40,0.40,0.40}{##1}}}
\expandafter\def\csname PY@tok@gp\endcsname{\let\PY@bf=\textbf\def\PY@tc##1{\textcolor[rgb]{0.00,0.00,0.50}{##1}}}
\expandafter\def\csname PY@tok@sh\endcsname{\def\PY@tc##1{\textcolor[rgb]{0.73,0.13,0.13}{##1}}}
\expandafter\def\csname PY@tok@ow\endcsname{\let\PY@bf=\textbf\def\PY@tc##1{\textcolor[rgb]{0.67,0.13,1.00}{##1}}}
\expandafter\def\csname PY@tok@sx\endcsname{\def\PY@tc##1{\textcolor[rgb]{0.00,0.50,0.00}{##1}}}
\expandafter\def\csname PY@tok@bp\endcsname{\def\PY@tc##1{\textcolor[rgb]{0.00,0.50,0.00}{##1}}}
\expandafter\def\csname PY@tok@c1\endcsname{\let\PY@it=\textit\def\PY@tc##1{\textcolor[rgb]{0.25,0.50,0.50}{##1}}}
\expandafter\def\csname PY@tok@o\endcsname{\def\PY@tc##1{\textcolor[rgb]{0.40,0.40,0.40}{##1}}}
\expandafter\def\csname PY@tok@kc\endcsname{\let\PY@bf=\textbf\def\PY@tc##1{\textcolor[rgb]{0.00,0.50,0.00}{##1}}}
\expandafter\def\csname PY@tok@c\endcsname{\let\PY@it=\textit\def\PY@tc##1{\textcolor[rgb]{0.25,0.50,0.50}{##1}}}
\expandafter\def\csname PY@tok@mf\endcsname{\def\PY@tc##1{\textcolor[rgb]{0.40,0.40,0.40}{##1}}}
\expandafter\def\csname PY@tok@err\endcsname{\def\PY@bc##1{\setlength{\fboxsep}{0pt}\fcolorbox[rgb]{1.00,0.00,0.00}{1,1,1}{\strut ##1}}}
\expandafter\def\csname PY@tok@mb\endcsname{\def\PY@tc##1{\textcolor[rgb]{0.40,0.40,0.40}{##1}}}
\expandafter\def\csname PY@tok@ss\endcsname{\def\PY@tc##1{\textcolor[rgb]{0.10,0.09,0.49}{##1}}}
\expandafter\def\csname PY@tok@sr\endcsname{\def\PY@tc##1{\textcolor[rgb]{0.73,0.40,0.53}{##1}}}
\expandafter\def\csname PY@tok@mo\endcsname{\def\PY@tc##1{\textcolor[rgb]{0.40,0.40,0.40}{##1}}}
\expandafter\def\csname PY@tok@kd\endcsname{\let\PY@bf=\textbf\def\PY@tc##1{\textcolor[rgb]{0.00,0.50,0.00}{##1}}}
\expandafter\def\csname PY@tok@mi\endcsname{\def\PY@tc##1{\textcolor[rgb]{0.40,0.40,0.40}{##1}}}
\expandafter\def\csname PY@tok@kn\endcsname{\let\PY@bf=\textbf\def\PY@tc##1{\textcolor[rgb]{0.00,0.50,0.00}{##1}}}
\expandafter\def\csname PY@tok@cpf\endcsname{\let\PY@it=\textit\def\PY@tc##1{\textcolor[rgb]{0.25,0.50,0.50}{##1}}}
\expandafter\def\csname PY@tok@kr\endcsname{\let\PY@bf=\textbf\def\PY@tc##1{\textcolor[rgb]{0.00,0.50,0.00}{##1}}}
\expandafter\def\csname PY@tok@s\endcsname{\def\PY@tc##1{\textcolor[rgb]{0.73,0.13,0.13}{##1}}}
\expandafter\def\csname PY@tok@kp\endcsname{\def\PY@tc##1{\textcolor[rgb]{0.00,0.50,0.00}{##1}}}
\expandafter\def\csname PY@tok@w\endcsname{\def\PY@tc##1{\textcolor[rgb]{0.73,0.73,0.73}{##1}}}
\expandafter\def\csname PY@tok@kt\endcsname{\def\PY@tc##1{\textcolor[rgb]{0.69,0.00,0.25}{##1}}}
\expandafter\def\csname PY@tok@sc\endcsname{\def\PY@tc##1{\textcolor[rgb]{0.73,0.13,0.13}{##1}}}
\expandafter\def\csname PY@tok@sb\endcsname{\def\PY@tc##1{\textcolor[rgb]{0.73,0.13,0.13}{##1}}}
\expandafter\def\csname PY@tok@k\endcsname{\let\PY@bf=\textbf\def\PY@tc##1{\textcolor[rgb]{0.00,0.50,0.00}{##1}}}
\expandafter\def\csname PY@tok@se\endcsname{\let\PY@bf=\textbf\def\PY@tc##1{\textcolor[rgb]{0.73,0.40,0.13}{##1}}}
\expandafter\def\csname PY@tok@sd\endcsname{\let\PY@it=\textit\def\PY@tc##1{\textcolor[rgb]{0.73,0.13,0.13}{##1}}}

\def\PYZbs{\char`\\}
\def\PYZus{\char`\_}
\def\PYZob{\char`\{}
\def\PYZcb{\char`\}}
\def\PYZca{\char`\^}
\def\PYZam{\char`\&}
\def\PYZlt{\char`\<}
\def\PYZgt{\char`\>}
\def\PYZsh{\char`\#}
\def\PYZpc{\char`\%}
\def\PYZdl{\char`\$}
\def\PYZhy{\char`\-}
\def\PYZsq{\char`\'}
\def\PYZdq{\char`\"}
\def\PYZti{\char`\~}
% for compatibility with earlier versions
\def\PYZat{@}
\def\PYZlb{[}
\def\PYZrb{]}
\makeatother


    % Exact colors from NB
    \definecolor{incolor}{rgb}{0.0, 0.0, 0.5}
    \definecolor{outcolor}{rgb}{0.545, 0.0, 0.0}



    
    % Prevent overflowing lines due to hard-to-break entities
    \sloppy 
    % Setup hyperref package
    \hypersetup{
      breaklinks=true,  % so long urls are correctly broken across lines
      colorlinks=true,
      urlcolor=blue,
      linkcolor=darkorange,
      citecolor=darkgreen,
      }
    % Slightly bigger margins than the latex defaults
    
    \geometry{verbose,tmargin=1in,bmargin=1in,lmargin=1in,rmargin=1in}
    
    

    \begin{document}
    
    
    \maketitle
    
    

    
    \section{Exercise Sheet 3: Sampling,
Simulation}\label{exercise-sheet-3-sampling-simulation}

    In this exercise sheet, we will simulate a Markov chain. In the first
part, we will consider a pure Python based implementation where a single
particle jumps from one position to another of the lattice, where all
transitions to neighboring states have the same probability. Then, we
will add probability weightings for the transitions. Finally, the
implementation will be parallelized to run many chains in parallel.

    \begin{Verbatim}[commandchars=\\\{\}]
{\color{incolor}In [{\color{incolor}1}]:} \PY{o}{\PYZpc{}}\PY{k}{matplotlib} inline
\end{Verbatim}

    \subsection{Exercise 1: Random moves in a lattice (20
P)}\label{exercise-1-random-moves-in-a-lattice-20-p}

In this exercise, we will simulate the propagation of particles in a
graph composed of 8 states (denoted by letters A-H) and stored in the
variable \texttt{S} defined in the cell below. The lattice is the
following:

\begin{figure}
\centering
\includegraphics{lattice.png}
\caption{}
\end{figure}

The particle starts in state \texttt{A} and then jumps randomly from its
current state to one of its neighbors, all with same probability. Note
that it cannot stay at the current position. The dictionary \texttt{T}
defined in the cell below encode such transition behavior.

    \begin{Verbatim}[commandchars=\\\{\}]
{\color{incolor}In [{\color{incolor}8}]:} \PY{c+c1}{\PYZsh{} List of states}
        \PY{n}{S} \PY{o}{=} \PY{l+s+s1}{\PYZsq{}}\PY{l+s+s1}{ABCDEFGH}\PY{l+s+s1}{\PYZsq{}}
        
        \PY{c+c1}{\PYZsh{} Set of transitions}
        \PY{n}{T} \PY{o}{=} \PY{p}{\PYZob{}}\PY{l+s+s1}{\PYZsq{}}\PY{l+s+s1}{A}\PY{l+s+s1}{\PYZsq{}}\PY{p}{:}\PY{l+s+s1}{\PYZsq{}}\PY{l+s+s1}{BE}\PY{l+s+s1}{\PYZsq{}}\PY{p}{,}\PY{l+s+s1}{\PYZsq{}}\PY{l+s+s1}{B}\PY{l+s+s1}{\PYZsq{}}\PY{p}{:}\PY{l+s+s1}{\PYZsq{}}\PY{l+s+s1}{AFC}\PY{l+s+s1}{\PYZsq{}}\PY{p}{,}\PY{l+s+s1}{\PYZsq{}}\PY{l+s+s1}{C}\PY{l+s+s1}{\PYZsq{}}\PY{p}{:}\PY{l+s+s1}{\PYZsq{}}\PY{l+s+s1}{BGD}\PY{l+s+s1}{\PYZsq{}}\PY{p}{,}\PY{l+s+s1}{\PYZsq{}}\PY{l+s+s1}{D}\PY{l+s+s1}{\PYZsq{}}\PY{p}{:}\PY{l+s+s1}{\PYZsq{}}\PY{l+s+s1}{CH}\PY{l+s+s1}{\PYZsq{}}\PY{p}{,}\PY{l+s+s1}{\PYZsq{}}\PY{l+s+s1}{E}\PY{l+s+s1}{\PYZsq{}}\PY{p}{:}\PY{l+s+s1}{\PYZsq{}}\PY{l+s+s1}{AF}\PY{l+s+s1}{\PYZsq{}}\PY{p}{,}\PY{l+s+s1}{\PYZsq{}}\PY{l+s+s1}{F}\PY{l+s+s1}{\PYZsq{}}\PY{p}{:}\PY{l+s+s1}{\PYZsq{}}\PY{l+s+s1}{EBG}\PY{l+s+s1}{\PYZsq{}}\PY{p}{,}\PY{l+s+s1}{\PYZsq{}}\PY{l+s+s1}{G}\PY{l+s+s1}{\PYZsq{}}\PY{p}{:}\PY{l+s+s1}{\PYZsq{}}\PY{l+s+s1}{FCH}\PY{l+s+s1}{\PYZsq{}}\PY{p}{,}\PY{l+s+s1}{\PYZsq{}}\PY{l+s+s1}{H}\PY{l+s+s1}{\PYZsq{}}\PY{p}{:}\PY{l+s+s1}{\PYZsq{}}\PY{l+s+s1}{GD}\PY{l+s+s1}{\PYZsq{}}\PY{p}{\PYZcb{}}
\end{Verbatim}

    \begin{Verbatim}[commandchars=\\\{\}]
{\color{incolor}In [{\color{incolor}9}]:} \PY{k+kn}{import} \PY{n+nn}{random}
        \PY{k}{def} \PY{n+nf}{fraction\PYZus{}of\PYZus{}time}\PY{p}{(}\PY{n}{S}\PY{p}{,} \PY{n}{T}\PY{p}{,} \PY{n}{iterations}\PY{p}{)}\PY{p}{:}
            \PY{n}{states} \PY{o}{=} \PY{n}{S}\PY{p}{[}\PY{l+m+mi}{0}\PY{p}{]}
            \PY{c+c1}{\PYZsh{} simulate the experiment and run it for 1999 iterations.}
            \PY{k}{for} \PY{n}{i} \PY{o+ow}{in} \PY{n+nb}{range}\PY{p}{(}\PY{n}{iterations}\PY{p}{)}\PY{p}{:}
                \PY{n}{next\PYZus{}state} \PY{o}{=} \PY{n}{random}\PY{o}{.}\PY{n}{choice}\PY{p}{(}\PY{n}{T}\PY{p}{[}\PY{n}{states}\PY{p}{[}\PY{n}{i}\PY{p}{]}\PY{p}{]}\PY{p}{)} \PY{c+c1}{\PYZsh{}state[i] is the current state}
                \PY{n}{states} \PY{o}{+}\PY{o}{=} \PY{n}{next\PYZus{}state}
            
            \PY{c+c1}{\PYZsh{} compute the fraction of the time of each state}
            \PY{n}{counter} \PY{o}{=} \PY{p}{\PYZob{}}\PY{n}{S}\PY{p}{[}\PY{n}{i}\PY{p}{]}\PY{p}{:}\PY{l+m+mf}{0.0} \PY{k}{for} \PY{n}{i} \PY{o+ow}{in} \PY{n+nb}{range}\PY{p}{(}\PY{n+nb}{len}\PY{p}{(}\PY{n}{S}\PY{p}{)}\PY{p}{)}\PY{p}{\PYZcb{}}
            \PY{k}{for} \PY{n}{i} \PY{o+ow}{in} \PY{n+nb}{range}\PY{p}{(}\PY{n+nb}{len}\PY{p}{(}\PY{n}{states}\PY{p}{)}\PY{p}{)}\PY{p}{:}
                \PY{n}{counter}\PY{p}{[}\PY{n}{states}\PY{p}{[}\PY{n}{i}\PY{p}{]}\PY{p}{]} \PY{o}{=} \PY{n}{counter}\PY{p}{[}\PY{n}{states}\PY{p}{[}\PY{n}{i}\PY{p}{]}\PY{p}{]} \PY{o}{+} \PY{l+m+mi}{1}
            \PY{n}{fractions} \PY{o}{=} \PY{n+nb}{list}\PY{p}{(}\PY{n+nb}{map}\PY{p}{(}\PY{k}{lambda} \PY{n}{x}\PY{p}{:} \PY{n}{x}\PY{p}{[}\PY{l+m+mi}{1}\PY{p}{]} \PY{o}{/} \PY{n+nb}{len}\PY{p}{(}\PY{n}{states}\PY{p}{)}\PY{p}{,} \PY{n}{counter}\PY{o}{.}\PY{n}{items}\PY{p}{(}\PY{p}{)}\PY{p}{)}\PY{p}{)}
            \PY{l+s+sd}{\PYZsq{}\PYZsq{}\PYZsq{}}
        \PY{l+s+sd}{    \PYZsh{} testing}
        \PY{l+s+sd}{    for i in counter.items():}
        \PY{l+s+sd}{        print(i)}
        \PY{l+s+sd}{    print(fractions)}
        \PY{l+s+sd}{    \PYZsq{}\PYZsq{}\PYZsq{}}
            \PY{k}{return} \PY{n}{states}\PY{p}{,} \PY{n}{fractions}
\end{Verbatim}

    Using pure Python, simulate the experiment below and run it for 1999
iterations. Print the sequence of first 400 states visited by the
particle. To obtain the same results as in \texttt{pdf} solution file,
you should initialize the seed of the module \texttt{random} to value
\texttt{123} using the function \texttt{random.seed} before starting the
simulation.

    \begin{Verbatim}[commandchars=\\\{\}]
{\color{incolor}In [{\color{incolor}12}]:} \PY{c+c1}{\PYZsh{}\PYZsh{}\PYZsh{} solutions.exercise1a()}
         \PY{c+c1}{\PYZsh{}random.seed(123)}
         \PY{n}{states}\PY{p}{,} \PY{n}{fractions} \PY{o}{=} \PY{n}{fraction\PYZus{}of\PYZus{}time}\PY{p}{(}\PY{n}{S}\PY{p}{,} \PY{n}{T}\PY{p}{,} \PY{l+m+mi}{2000}\PY{p}{)}
         \PY{c+c1}{\PYZsh{} print the first 400 states of the simulation}
         \PY{k}{print}\PY{p}{(}\PY{n}{states}\PY{p}{[}\PY{l+m+mi}{0}\PY{p}{:}\PY{l+m+mi}{100}\PY{p}{]}\PY{p}{)}
         \PY{k}{print}\PY{p}{(}\PY{n}{states}\PY{p}{[}\PY{l+m+mi}{100}\PY{p}{:}\PY{l+m+mi}{200}\PY{p}{]}\PY{p}{)}
         \PY{k}{print}\PY{p}{(}\PY{n}{states}\PY{p}{[}\PY{l+m+mi}{200}\PY{p}{:}\PY{l+m+mi}{300}\PY{p}{]}\PY{p}{)}
         \PY{k}{print}\PY{p}{(}\PY{n}{states}\PY{p}{[}\PY{l+m+mi}{300}\PY{p}{:}\PY{l+m+mi}{400}\PY{p}{]}\PY{p}{)}
\end{Verbatim}

    \begin{Verbatim}[commandchars=\\\{\}]
AEFEFGFGCBFBCBCDCGHDHDHGCGCBABCGCDCDHDCBCDHGFBFGHDHGHDCGFBABCBCBCBFBAEFBFBCGCBFBFBCGFBCGHGCGFGFEAEFB
CGFBCDCDHGHGFBCDHGHDHGHDHDHGFBFEFBCBAEFGCDCGCDHDHGHGHGHDCGFGFEABABCGCBFGCDCDHGFBFBAEFBFBCDCGHGCDCDCG
CBABABCGFEAEFBFEABCGFGFEFEFBFBABFBCGCBABABFGCGFGHDHDHDHDCDCBABFBCBCGCDHGFEFGCBCBCDHGHGCDCDCBABCBFBFB
CDHDCGHDCDCDHGHDHDCDCBCGHGCBFGCBAEAEAEFBCDCGCBCGHDHDCDHDCGFGHDCBAEFEFBCDCBCBCGFBAEAEFBAEFBFGCDCBABAB
    \end{Verbatim}

    Using \texttt{matplotlib}, produce a bar plot
(\texttt{matplotlib.pyplot.bar}) showing the fraction of the time the
particle is found in a given state, averaged over the whole simulation.

    \begin{Verbatim}[commandchars=\\\{\}]
{\color{incolor}In [{\color{incolor}13}]:} \PY{c+c1}{\PYZsh{}\PYZsh{}\PYZsh{} solutions.exercise1b()}
         \PY{k+kn}{import} \PY{n+nn}{matplotlib}
         \PY{k+kn}{from} \PY{n+nn}{matplotlib} \PY{k+kn}{import} \PY{n}{pyplot} \PY{k}{as} \PY{n}{plt}
         \PY{k+kn}{from} \PY{n+nn}{IPython.display} \PY{k+kn}{import} \PY{n}{set\PYZus{}matplotlib\PYZus{}formats}
         \PY{n}{set\PYZus{}matplotlib\PYZus{}formats}\PY{p}{(}\PY{l+s+s1}{\PYZsq{}}\PY{l+s+s1}{pdf}\PY{l+s+s1}{\PYZsq{}}\PY{p}{,} \PY{l+s+s1}{\PYZsq{}}\PY{l+s+s1}{png}\PY{l+s+s1}{\PYZsq{}}\PY{p}{)}
         \PY{n}{plt}\PY{o}{.}\PY{n}{rcParams}\PY{p}{[}\PY{l+s+s1}{\PYZsq{}}\PY{l+s+s1}{savefig.dpi}\PY{l+s+s1}{\PYZsq{}}\PY{p}{]} \PY{o}{=} \PY{l+m+mi}{90}
         
         \PY{n}{s\PYZus{}list} \PY{o}{=} \PY{p}{[}\PY{n}{S}\PY{p}{[}\PY{n}{i}\PY{p}{]} \PY{k}{for} \PY{n}{i} \PY{o+ow}{in} \PY{n+nb}{range}\PY{p}{(}\PY{n+nb}{len}\PY{p}{(}\PY{n}{S}\PY{p}{)}\PY{p}{)}\PY{p}{]}
         \PY{n}{matplotlib}\PY{o}{.}\PY{n}{pyplot}\PY{o}{.}\PY{n}{bar}\PY{p}{(}\PY{n}{s\PYZus{}list}\PY{p}{,} \PY{n}{fractions}\PY{p}{,} \PY{n}{align}\PY{o}{=}\PY{l+s+s1}{\PYZsq{}}\PY{l+s+s1}{center}\PY{l+s+s1}{\PYZsq{}}\PY{p}{,} \PY{n}{color}\PY{o}{=}\PY{l+s+s1}{\PYZsq{}}\PY{l+s+s1}{blue}\PY{l+s+s1}{\PYZsq{}}\PY{p}{)}
\end{Verbatim}

            \begin{Verbatim}[commandchars=\\\{\}]
{\color{outcolor}Out[{\color{outcolor}13}]:} <Container object of 8 artists>
\end{Verbatim}
        
    \begin{center}
    \adjustimage{max size={0.9\linewidth}{0.9\paperheight}}{sheet3_files/sheet3_9_1.pdf}
    \end{center}
    { \hspace*{\fill} \\}
    
    \subsection{Exercise 2: Adding a special state (20
P)}\label{exercise-2-adding-a-special-state-20-p}

Suppose now that the rule is modified such that everytime the particle
is in state \texttt{F}, it always moves to \texttt{E} in the next step.

\begin{itemize}
\tightlist
\item
  Modify the code to handle this special case, and create a bar plot for
  the new states distribution.
\end{itemize}

    \begin{Verbatim}[commandchars=\\\{\}]
{\color{incolor}In [{\color{incolor}14}]:} \PY{c+c1}{\PYZsh{}\PYZsh{}\PYZsh{} solutions.exercise2()}
         \PY{n}{T\PYZus{}modified} \PY{o}{=} \PY{p}{\PYZob{}}\PY{l+s+s1}{\PYZsq{}}\PY{l+s+s1}{A}\PY{l+s+s1}{\PYZsq{}}\PY{p}{:}\PY{l+s+s1}{\PYZsq{}}\PY{l+s+s1}{BE}\PY{l+s+s1}{\PYZsq{}}\PY{p}{,}\PY{l+s+s1}{\PYZsq{}}\PY{l+s+s1}{B}\PY{l+s+s1}{\PYZsq{}}\PY{p}{:}\PY{l+s+s1}{\PYZsq{}}\PY{l+s+s1}{AFC}\PY{l+s+s1}{\PYZsq{}}\PY{p}{,}\PY{l+s+s1}{\PYZsq{}}\PY{l+s+s1}{C}\PY{l+s+s1}{\PYZsq{}}\PY{p}{:}\PY{l+s+s1}{\PYZsq{}}\PY{l+s+s1}{BGD}\PY{l+s+s1}{\PYZsq{}}\PY{p}{,}\PY{l+s+s1}{\PYZsq{}}\PY{l+s+s1}{D}\PY{l+s+s1}{\PYZsq{}}\PY{p}{:}\PY{l+s+s1}{\PYZsq{}}\PY{l+s+s1}{CH}\PY{l+s+s1}{\PYZsq{}}\PY{p}{,}\PY{l+s+s1}{\PYZsq{}}\PY{l+s+s1}{E}\PY{l+s+s1}{\PYZsq{}}\PY{p}{:}\PY{l+s+s1}{\PYZsq{}}\PY{l+s+s1}{AF}\PY{l+s+s1}{\PYZsq{}}\PY{p}{,}\PY{l+s+s1}{\PYZsq{}}\PY{l+s+s1}{F}\PY{l+s+s1}{\PYZsq{}}\PY{p}{:}\PY{l+s+s1}{\PYZsq{}}\PY{l+s+s1}{E}\PY{l+s+s1}{\PYZsq{}}\PY{p}{,}\PY{l+s+s1}{\PYZsq{}}\PY{l+s+s1}{G}\PY{l+s+s1}{\PYZsq{}}\PY{p}{:}\PY{l+s+s1}{\PYZsq{}}\PY{l+s+s1}{FCH}\PY{l+s+s1}{\PYZsq{}}\PY{p}{,}\PY{l+s+s1}{\PYZsq{}}\PY{l+s+s1}{H}\PY{l+s+s1}{\PYZsq{}}\PY{p}{:}\PY{l+s+s1}{\PYZsq{}}\PY{l+s+s1}{GD}\PY{l+s+s1}{\PYZsq{}}\PY{p}{\PYZcb{}}
         \PY{n}{states2}\PY{p}{,} \PY{n}{fractions2} \PY{o}{=} \PY{n}{fraction\PYZus{}of\PYZus{}time}\PY{p}{(}\PY{n}{S}\PY{p}{,} \PY{n}{T\PYZus{}modified}\PY{p}{,} \PY{l+m+mi}{2000}\PY{p}{)}
         \PY{n}{matplotlib}\PY{o}{.}\PY{n}{pyplot}\PY{o}{.}\PY{n}{bar}\PY{p}{(}\PY{n}{s\PYZus{}list}\PY{p}{,} \PY{n}{fractions2}\PY{p}{,} \PY{n}{align}\PY{o}{=}\PY{l+s+s1}{\PYZsq{}}\PY{l+s+s1}{center}\PY{l+s+s1}{\PYZsq{}}\PY{p}{,} \PY{n}{color}\PY{o}{=}\PY{l+s+s1}{\PYZsq{}}\PY{l+s+s1}{blue}\PY{l+s+s1}{\PYZsq{}}\PY{p}{)}
\end{Verbatim}

            \begin{Verbatim}[commandchars=\\\{\}]
{\color{outcolor}Out[{\color{outcolor}14}]:} <Container object of 8 artists>
\end{Verbatim}
        
    \begin{center}
    \adjustimage{max size={0.9\linewidth}{0.9\paperheight}}{sheet3_files/sheet3_11_1.pdf}
    \end{center}
    { \hspace*{\fill} \\}
    
    \subsection{Exercise 3: Exact solution to the previous exercise (20
P)}\label{exercise-3-exact-solution-to-the-previous-exercise-20-p}

For simple Markov chains, a number of statistics can be obtained
analytically from the structure of the transition model, in particular,
by analysis of the transition matrix.

\begin{itemize}
\tightlist
\item
  Compute the transition matrices associated to the models of exercise 1
  and 2 (make sure that each row in these matrices sums to 1).
\item
  Give the transition matrices as argument to the function
  \texttt{utils.getstationary(P)} and print their result.
\end{itemize}

This last function computes in closed form the stationary distribution
associated to a given transition matrix \texttt{P} (i.e.~the one we
would get if running the simulation with such transition matrix for
infinitely many time steps and looking at state frequencies).

    \begin{Verbatim}[commandchars=\\\{\}]
{\color{incolor}In [{\color{incolor}17}]:} \PY{k+kn}{import} \PY{n+nn}{numpy}
         \PY{k}{def} \PY{n+nf}{compute\PYZus{}trasition\PYZus{}matrix}\PY{p}{(}\PY{n}{states}\PY{p}{,} \PY{n}{transition}\PY{p}{)}\PY{p}{:}
             \PY{n}{t} \PY{o}{=} \PY{n}{numpy}\PY{o}{.}\PY{n}{zeros}\PY{p}{(}\PY{p}{[}\PY{l+m+mi}{8}\PY{p}{,}\PY{l+m+mi}{8}\PY{p}{]}\PY{p}{)}
             \PY{c+c1}{\PYZsh{}mapping = \PYZob{}\PYZsq{}A\PYZsq{}: 0, \PYZsq{}B\PYZsq{}: 1, \PYZsq{}C\PYZsq{}: 2, \PYZsq{}D\PYZsq{}: 3, \PYZsq{}E\PYZsq{}: 4, \PYZsq{}F\PYZsq{}: 5, \PYZsq{}G\PYZsq{}: 6, \PYZsq{}H\PYZsq{}: 7\PYZcb{}}
             \PY{n}{mapping} \PY{o}{=} \PY{p}{\PYZob{}}\PY{n}{states}\PY{p}{[}\PY{n}{i}\PY{p}{]}\PY{p}{:}\PY{n}{i} \PY{k}{for} \PY{n}{i} \PY{o+ow}{in} \PY{n+nb}{range}\PY{p}{(}\PY{n+nb}{len}\PY{p}{(}\PY{n}{states}\PY{p}{)}\PY{p}{)}\PY{p}{\PYZcb{}}
             \PY{k}{for} \PY{n}{i} \PY{o+ow}{in} \PY{n}{states}\PY{p}{:}
                 \PY{n}{val} \PY{o}{=} \PY{l+m+mf}{1.0} \PY{o}{/} \PY{n+nb}{len}\PY{p}{(}\PY{n}{transition}\PY{p}{[}\PY{n}{i}\PY{p}{]}\PY{p}{)}
                 \PY{k}{for} \PY{n}{j} \PY{o+ow}{in} \PY{n}{transition}\PY{p}{[}\PY{n}{i}\PY{p}{]}\PY{p}{:}
                     \PY{n}{t}\PY{p}{[}\PY{n}{mapping}\PY{p}{[}\PY{n}{i}\PY{p}{]}\PY{p}{,} \PY{n}{mapping}\PY{p}{[}\PY{n}{j}\PY{p}{]}\PY{p}{]} \PY{o}{=} \PY{n}{val}
             \PY{c+c1}{\PYZsh{}print(t)}
             \PY{k}{return} \PY{n}{t}
\end{Verbatim}

    \begin{Verbatim}[commandchars=\\\{\}]
{\color{incolor}In [{\color{incolor}47}]:} \PY{c+c1}{\PYZsh{}\PYZsh{}\PYZsh{} solutions.exercise3()}
         \PY{c+c1}{\PYZsh{}A0 B1 C2 D3 E4 F5 G6 H7}
         \PY{c+c1}{\PYZsh{}T = \PYZob{}\PYZsq{}A\PYZsq{}:\PYZsq{}BE\PYZsq{},\PYZsq{}B\PYZsq{}:\PYZsq{}AFC\PYZsq{},\PYZsq{}C\PYZsq{}:\PYZsq{}BGD\PYZsq{},\PYZsq{}D\PYZsq{}:\PYZsq{}CH\PYZsq{},\PYZsq{}E\PYZsq{}:\PYZsq{}AF\PYZsq{},\PYZsq{}F\PYZsq{}:\PYZsq{}EBG\PYZsq{},\PYZsq{}G\PYZsq{}:\PYZsq{}FCH\PYZsq{},\PYZsq{}H\PYZsq{}:\PYZsq{}GD\PYZsq{}\PYZcb{}}
         \PY{c+c1}{\PYZsh{}T\PYZus{}modified = \PYZob{}\PYZsq{}A\PYZsq{}:\PYZsq{}BE\PYZsq{},\PYZsq{}B\PYZsq{}:\PYZsq{}AFC\PYZsq{},\PYZsq{}C\PYZsq{}:\PYZsq{}BGD\PYZsq{},\PYZsq{}D\PYZsq{}:\PYZsq{}CH\PYZsq{},\PYZsq{}E\PYZsq{}:\PYZsq{}AF\PYZsq{},\PYZsq{}F\PYZsq{}:\PYZsq{}E\PYZsq{},\PYZsq{}G\PYZsq{}:\PYZsq{}FCH\PYZsq{},\PYZsq{}H\PYZsq{}:\PYZsq{}GD\PYZsq{}\PYZcb{}}
         \PY{k+kn}{import} \PY{n+nn}{utils}
         
         \PY{n}{t\PYZus{}matrix1} \PY{o}{=} \PY{n}{compute\PYZus{}trasition\PYZus{}matrix}\PY{p}{(}\PY{n}{S}\PY{p}{,} \PY{n}{T}\PY{p}{)}
         \PY{n}{result} \PY{o}{=} \PY{n}{utils}\PY{o}{.}\PY{n}{getstationary}\PY{p}{(}\PY{n}{t\PYZus{}matrix1}\PY{p}{)}
         \PY{k}{print}\PY{p}{(}\PY{l+s+s2}{\PYZdq{}}\PY{l+s+s2}{Exercise 1 [}\PY{l+s+s2}{\PYZdq{}}\PY{p}{,} \PY{n}{end}\PY{o}{=}\PY{l+s+s1}{\PYZsq{}}\PY{l+s+s1}{ }\PY{l+s+s1}{\PYZsq{}}\PY{p}{)}
         \PY{k}{for} \PY{n}{i} \PY{o+ow}{in} \PY{n}{result}\PY{p}{:}
             \PY{k}{print}\PY{p}{(}\PY{l+s+s2}{\PYZdq{}}\PY{l+s+s2}{\PYZob{}:4.2f\PYZcb{}}\PY{l+s+s2}{\PYZdq{}}\PY{o}{.}\PY{n}{format}\PY{p}{(}\PY{n}{i}\PY{p}{)}\PY{p}{,} \PY{n}{end}\PY{o}{=}\PY{l+s+s1}{\PYZsq{}}\PY{l+s+s1}{ }\PY{l+s+s1}{\PYZsq{}}\PY{p}{)}  \PY{c+c1}{\PYZsh{}format specifier usage \PYZdq{}\PYZob{}[tuple\PYZus{}index]:[width][.precision][type]\PYZcb{}\PYZdq{}}
         \PY{k}{print}\PY{p}{(}\PY{l+s+s2}{\PYZdq{}}\PY{l+s+s2}{]}\PY{l+s+s2}{\PYZdq{}}\PY{p}{)}
         
         \PY{n}{t\PYZus{}matrix2} \PY{o}{=} \PY{n}{compute\PYZus{}trasition\PYZus{}matrix}\PY{p}{(}\PY{n}{S}\PY{p}{,} \PY{n}{T\PYZus{}modified}\PY{p}{)}
         \PY{n}{result} \PY{o}{=} \PY{n}{utils}\PY{o}{.}\PY{n}{getstationary}\PY{p}{(}\PY{n}{t\PYZus{}matrix2}\PY{p}{)}
         \PY{k}{print}\PY{p}{(}\PY{l+s+s2}{\PYZdq{}}\PY{l+s+s2}{Exercise 2 [}\PY{l+s+s2}{\PYZdq{}}\PY{p}{,}\PY{n}{end}\PY{o}{=}\PY{l+s+s1}{\PYZsq{}}\PY{l+s+s1}{ }\PY{l+s+s1}{\PYZsq{}}\PY{p}{)}
         \PY{k}{for} \PY{n}{i} \PY{o+ow}{in} \PY{n}{result}\PY{p}{:}
             \PY{k}{print}\PY{p}{(}\PY{l+s+s2}{\PYZdq{}}\PY{l+s+s2}{\PYZob{}:4.2f\PYZcb{}}\PY{l+s+s2}{\PYZdq{}}\PY{o}{.}\PY{n}{format}\PY{p}{(}\PY{n}{i}\PY{p}{)}\PY{p}{,} \PY{n}{end}\PY{o}{=}\PY{l+s+s1}{\PYZsq{}}\PY{l+s+s1}{ }\PY{l+s+s1}{\PYZsq{}}\PY{p}{)}
         \PY{k}{print}\PY{p}{(}\PY{l+s+s2}{\PYZdq{}}\PY{l+s+s2}{]}\PY{l+s+s2}{\PYZdq{}}\PY{p}{)}
\end{Verbatim}

    \begin{Verbatim}[commandchars=\\\{\}]
Exercise 1 [ 0.10 0.15 0.15 0.10 0.10 0.15 0.15 0.10 ]
Exercise 2 [ 0.19 0.12 0.08 0.04 0.29 0.20 0.04 0.04 ]
    \end{Verbatim}

    \subsection{Exercise 4: Adding non-uniform transition probabilities (20
P)}\label{exercise-4-adding-non-uniform-transition-probabilities-20-p}

We consider the original lattice defined by the variable \texttt{T}. We
set transition probabilities for each state to be such that: (1) the
probability of moving left is always twice the probability of moving
right when both moves are available. (2) The probability of moving
vertically is the same as the probability of moving horizontally.

\begin{itemize}
\tightlist
\item
  Build the transition matrix \texttt{P} implementing the described
  behavior, and compute its stationary distribution using the function
  \texttt{utils.getstationary(P)}.
\end{itemize}

(Hints: You can notice that for each state, the transitions towards
other states are always listed from left to right in the dictionary
\texttt{T}. Also note that characters A-H can be mapped to integer
values using the Python function ord(), thus, giving a direct relation
between state names and indices of the transition matrix.)

    \begin{Verbatim}[commandchars=\\\{\}]
{\color{incolor}In [{\color{incolor}59}]:} \PY{c+c1}{\PYZsh{}\PYZsh{}\PYZsh{} solutions.exercise4()}
         \PY{c+c1}{\PYZsh{}\PYZsh{}\PYZsh{} T = \PYZob{}\PYZsq{}A\PYZsq{}:\PYZsq{}BE\PYZsq{},\PYZsq{}B\PYZsq{}:\PYZsq{}AFC\PYZsq{},\PYZsq{}C\PYZsq{}:\PYZsq{}BGD\PYZsq{},\PYZsq{}D\PYZsq{}:\PYZsq{}CH\PYZsq{},\PYZsq{}E\PYZsq{}:\PYZsq{}AF\PYZsq{},\PYZsq{}F\PYZsq{}:\PYZsq{}EBG\PYZsq{},\PYZsq{}G\PYZsq{}:\PYZsq{}FCH\PYZsq{},\PYZsq{}H\PYZsq{}:\PYZsq{}GD\PYZsq{}\PYZcb{}}
         \PY{n}{t} \PY{o}{=} \PY{n}{numpy}\PY{o}{.}\PY{n}{zeros}\PY{p}{(}\PY{p}{[}\PY{l+m+mi}{8}\PY{p}{,}\PY{l+m+mi}{8}\PY{p}{]}\PY{p}{)}
         \PY{c+c1}{\PYZsh{}mapping = \PYZob{}states[i]:i for i in range(len(states))\PYZcb{} \PYZsh{}use ord() instead}
         \PY{k}{for} \PY{n}{i} \PY{o+ow}{in} \PY{n}{S}\PY{p}{:}
             \PY{k}{if} \PY{n}{i} \PY{o+ow}{in} \PY{l+s+s2}{\PYZdq{}}\PY{l+s+s2}{AEDH}\PY{l+s+s2}{\PYZdq{}}\PY{p}{:}
                 \PY{n}{val} \PY{o}{=} \PY{l+m+mf}{1.0} \PY{o}{/} \PY{n+nb}{len}\PY{p}{(}\PY{n}{T}\PY{p}{[}\PY{n}{i}\PY{p}{]}\PY{p}{)}
                 \PY{k}{for} \PY{n}{j} \PY{o+ow}{in} \PY{n}{T}\PY{p}{[}\PY{n}{i}\PY{p}{]}\PY{p}{:}
                     \PY{n}{t}\PY{p}{[}\PY{n+nb}{ord}\PY{p}{(}\PY{n}{i}\PY{p}{)}\PY{o}{\PYZhy{}}\PY{n+nb}{ord}\PY{p}{(}\PY{l+s+s1}{\PYZsq{}}\PY{l+s+s1}{A}\PY{l+s+s1}{\PYZsq{}}\PY{p}{)}\PY{p}{,} \PY{n+nb}{ord}\PY{p}{(}\PY{n}{j}\PY{p}{)}\PY{o}{\PYZhy{}}\PY{n+nb}{ord}\PY{p}{(}\PY{l+s+s1}{\PYZsq{}}\PY{l+s+s1}{A}\PY{l+s+s1}{\PYZsq{}}\PY{p}{)}\PY{p}{]} \PY{o}{=} \PY{n}{val}
             \PY{k}{else}\PY{p}{:}
                 \PY{n}{val} \PY{o}{=} \PY{p}{[}\PY{l+m+mi}{1}\PY{o}{/}\PY{l+m+mi}{3}\PY{p}{,} \PY{l+m+mi}{1}\PY{o}{/}\PY{l+m+mi}{2}\PY{p}{,} \PY{l+m+mi}{1}\PY{o}{/}\PY{l+m+mi}{6}\PY{p}{]}
                 \PY{k}{for} \PY{n}{j} \PY{o+ow}{in} \PY{n+nb}{zip}\PY{p}{(}\PY{n}{T}\PY{p}{[}\PY{n}{i}\PY{p}{]}\PY{p}{,} \PY{n}{val}\PY{p}{)}\PY{p}{:}
                     \PY{n}{t}\PY{p}{[}\PY{n+nb}{ord}\PY{p}{(}\PY{n}{i}\PY{p}{)}\PY{o}{\PYZhy{}}\PY{n+nb}{ord}\PY{p}{(}\PY{l+s+s1}{\PYZsq{}}\PY{l+s+s1}{A}\PY{l+s+s1}{\PYZsq{}}\PY{p}{)}\PY{p}{,} \PY{n+nb}{ord}\PY{p}{(}\PY{n}{j}\PY{p}{[}\PY{l+m+mi}{0}\PY{p}{]}\PY{p}{)}\PY{o}{\PYZhy{}}\PY{n+nb}{ord}\PY{p}{(}\PY{l+s+s1}{\PYZsq{}}\PY{l+s+s1}{A}\PY{l+s+s1}{\PYZsq{}}\PY{p}{)}\PY{p}{]} \PY{o}{=} \PY{n}{j}\PY{p}{[}\PY{l+m+mi}{1}\PY{p}{]}
         \PY{c+c1}{\PYZsh{}print(t)}
         \PY{n}{result} \PY{o}{=} \PY{n}{utils}\PY{o}{.}\PY{n}{getstationary}\PY{p}{(}\PY{n}{t}\PY{p}{)}
         \PY{k}{print}\PY{p}{(}\PY{l+s+s2}{\PYZdq{}}\PY{l+s+s2}{[}\PY{l+s+s2}{\PYZdq{}}\PY{p}{,}\PY{n}{end}\PY{o}{=}\PY{l+s+s1}{\PYZsq{}}\PY{l+s+s1}{\PYZsq{}}\PY{p}{)}
         \PY{k}{for} \PY{n}{i} \PY{o+ow}{in} \PY{n}{result}\PY{p}{:}
             \PY{k}{print}\PY{p}{(}\PY{l+s+s2}{\PYZdq{}}\PY{l+s+s2}{\PYZob{}:5.2f\PYZcb{}}\PY{l+s+s2}{\PYZdq{}}\PY{o}{.}\PY{n}{format}\PY{p}{(}\PY{n}{i}\PY{p}{)}\PY{p}{,} \PY{n}{end}\PY{o}{=}\PY{l+s+s1}{\PYZsq{}}\PY{l+s+s1}{ }\PY{l+s+s1}{\PYZsq{}}\PY{p}{)}
         \PY{k}{print}\PY{p}{(}\PY{l+s+s2}{\PYZdq{}}\PY{l+s+s2}{]}\PY{l+s+s2}{\PYZdq{}}\PY{p}{)}
\end{Verbatim}

    \begin{Verbatim}[commandchars=\\\{\}]
[ 0.14  0.21  0.11  0.04  0.14  0.21  0.11  0.04 ]
    \end{Verbatim}

    \subsection{Exercise 5: Simulation for multiple particles (20
P)}\label{exercise-5-simulation-for-multiple-particles-20-p}

We let 1000 particles evolve simultaneously in the system described in
Exercise 4. The initial state of these particles is pseudo-random and
given by the function \texttt{utils.getinitialstate()}.

\begin{itemize}
\tightlist
\item
  Using the function \texttt{utils.mcstep()} that was introduced during
  the lecture, simulate this system for 500 time steps.
\item
  Estimate the stationary distribution by looking at the distribution of
  these particles in state space after 500 time steps.
\end{itemize}

For reproducibility, give seed values to the function utils.mcstep
corresponding to the current time step of the simulation (i.e.~from 0 to
499).

    \begin{Verbatim}[commandchars=\\\{\}]
{\color{incolor}In [{\color{incolor} }]:} \PY{c+c1}{\PYZsh{}\PYZsh{}\PYZsh{} Replace by your own code}
        \PY{k+kn}{import} \PY{n+nn}{solutions}
        \PY{n}{solutions}\PY{o}{.}\PY{n}{exercise5}\PY{p}{(}\PY{p}{)}
        \PY{c+c1}{\PYZsh{}\PYZsh{}\PYZsh{}}
\end{Verbatim}


    % Add a bibliography block to the postdoc
    
    
    
    \end{document}
